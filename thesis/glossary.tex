\newglossaryentry{cbt}	{
	name ={CBT},
	description={Computer Based Training, synonym for CAI.}
}

\newglossaryentry{cai}	{
	name={CAI},  
	description={Computer Aided Instruction. A form of electronic learning.}
}

\newglossaryentry{its}	{
	name={ITS},  
	plural= {ITSs},
	description={An Intelligent Tutor System uses intelligent technology to give feedback and hints during exercise execution.}
}

\newglossaryentry{slip}	{
	name={Slip},  
	text={slip},
	description={A student made an error although he understands the correct solution. }
}

\newglossaryentry{lucky}	{
	name={Lucky guess},  
	text={lucky guess},  
	plural={lucky guesses},
	description={A student does not know the correct answer, but picked the right solution anyway}
}

\newglossaryentry{constraintbased} {
	name={Constraint based tutor},
	description={An intelligent tutor that gives feedback based on constraints on the solution. 
	For example an SQL tutor will require that a solution is syntactically correct and yield the right results.}
}

\newglossaryentry{modeltracing} {
	name={Model tracing tutor},
	text={model tracing tutor},
	description={A tutor that gives feedback by tracing student progress against an expert solution.
	The tutor recognises not only mistakes, but also deviations from the solution strategy. }
}

\newglossaryentry{kst} {
	name={Knowlegde space theory},
	text={KST},
	description={A psychometric method for question selection based on prerequisite relations between items. }
}

\newglossaryentry{domainreasoner}{
	name={Domain reasoner},
	text={domain reasoner},
	description={A domain reasoner can reason about student solutions in a particular domain. 
	It can diagnose errors, evaluate if the student is still on the right track, or advice on a next step to take.}
}
\newglossaryentry{studentmodel}{
	name={Student model},
	text={student model},
	description={A part in the ITS where information about the student, such as learning progress and preferences, is maintained. 
	}
}
\newglossaryentry{innerloop}{
	name={Inner loop},
	text={innerloop},
	description={The innerloop uses intelligent technology to coach a student through an exercise. It works on the task step level.}
}
\newglossaryentry{outerloop}{
	name={Outer loop},
	text={student model},
	description={The outer loop is responsible for selecting exercises and learning tasks. 
	It can vary from a menu where the student can choose any exercise to an advanced recommendation system.}
}

\newglossaryentry{actr}{
	name={ACT-R},
	text={ACT-R},
	description={(Adaptive Character of Thought). A theory oflearning and cognition based on a theory of the mind containing a large number of rules.}
}

\newglossaryentry{askelle}{
	name={Ask-Elle},
	text={Ask-Elle},
	description={A tutor based on an IDEAS domain reasoner for functional programming in Haskell.}
}
\newglossaryentry{hee}{
	name={HEE},
	text={HEE},
	description={Haskell Expression Evaluator, a tutor that conducts exercises where functional programs are evaluated as expressions.}
}

\newglossaryentry{sap}{
	name={SAP},
	text={SAP},
	plural={SAPs},
	description={A Systematic Approach to Problem Solving.}
}

\newglossaryentry{madapt}{
	name={macroadaptation},
	text={macroadaptation},
	description={An outer loop mechanism where tasks are selected that contain only one or two new knowledge components.}
}


\newglossaryentry{loc}{
	name={LOC},
	text={LOC},
	description={Lines of code. Different interpretations are possible with regard to empty lines and comment lines.}
}



\newglossaryentry{tensteps}{
	name={TenSteps methodology},
	text={TenSteps  methodology},
	plural={TenSteps},
	description={The Ten Steps/Four components instructional design methodology}
}













