\chapter{Introduction}
The internet has changed the pace of development in many fields. 
Working with constantly changing technology puts pressure on professionals to keep up to date.
Knowledge workers need updates on skills every few years and lifelong learning has become a fact of life \citep{lifelonglearning}.
E-learning is a way to meet the growing demand for training and education.
E-learning solutions in the form of Computer Based Training (\gls{cbt}) or Computer Aided instruction (\gls{cai}) have existed for several decades \citep{woolf_2010}.

An Intelligent Tutoring System (\gls{its}) is an E-learning system in which intelligent technology is used to support working on exercises.
The tutor gives hints and feedback on steps during the exercises.
Intelligent Tutors have been built for several domains,  in mathematics, physics, programming in Lisp  \citep{tutor-pat, tutor-andes, anderson1986}, etc. 

\Glspl{domainreasoner} from the  IDEAS\footnote{\url{http://ideas.cs.uu.nl}}  project are an intelligent technology that can be used in an \gls{its}.
These \glspl{domainreasoner} can evaluate if the student is still on the right track with an exercise, diagnose an error or give hints which step to take if the student is stuck.
The Ask-Elle Tutor uses domain reasoners to support students with functional programming  exercises in the Haskell programming language \citep{Gerdes2017}. 


These \glspl{domainreasoner} have a stateless architecture. 
However, an \gls{its} must keep state to keep track of a students progress with exercises, and knowledge gained.
The \gls{studentmodel} is the part that keeps information per student.
Different students may have different learning behaviour, and some need more exercises than others. 
The \gls{studentmodel} makes the \gls{its} adaptive to such needs.

The Ten steps/Four component is an instructional design methodology for learning complex skills \citep{kirschner_2013}.
In this research we investigate how a student model can be designed for \gls{its} with domain reasoners from the IDEAS project, following the \gls{tensteps}.



\section{Prior work}
This research is based upon work from different research areas, and aims at integrating concepts from these areas.

\subsection{Intelligent Tutor Systems and student models}
Intelligent Tutors have existed for several decades \citep{woolf_2010}. 
One of the first tutors in the area of programming is the Lisp tutor \citep{anderson1986}.

\glspl{its} maintain data on student progress in a student model. 
The general principles in student modelling were described by \citet{studentModelling} and  \citet{brusilovskiy_1993, brusilovsky_1994}.
Reasoning about student knowledge is difficult as we cannot observe student knowledge directly.
We can collect evidence of student knowledge by observing external behavior.
This may be difficult to interpret. For example, when answering a multiple choice question the student can make a \gls{lucky} or a \gls{slip}.

Many systems use statistical methods for inferencing in the student model. 
Knowledge tracing is a statistical inferencing method based on Bayesian statistics   \citep{corbett_1995}.
Knowledge spaces theory (\gls{kst}) was developed to improve assessment methods  \citep{doignon_falmagne_2016}. 
\gls{kst} is based on knowledge states that have prerequisite relationships, for efficient selection of questions in an assessment.
A student model can use the same principles for selecting exercises, hints or feedback.

Many researchers use semantic web technology to build the student model as ontology \citep{ontology}.
Semantic web technology uses description logic for reasoning with the open world principle \citep{dl2007}. 
The open world principle assumes that it must reason with incomplete information.
This contrasts with the closed world reasoning were it is assumed that all information is available.
Students often acquire much knowledge outside the \gls{its}, e.g. by searching the internet or reading books. 
When the \gls{its} has incomplete information on student knowledge, an open world reasoner is needed.


Student Modelling is a subtopic in the area of User Modelling.
A user model keeps track of the users preferences and cognitive states, that are derived from the interaction history.
The user model is used to make software adaptive to user needs by generating help or recommendations, for example in web shops, wordprocessors or music players.
User modelling in intelligent user interfaces is an active research area \citep{Brusilovsky2007}.

Student modelling is a large research area.
Chrysafiadi and Virvou performed a literature review on approaches for student modelling \citep{chrysafiadi_2013}.
VanLehn described the behaviour of Tutoring systems in terms of an inner and outer loop.
The inner loop uses intelligent technology for conducting an exercise.
The outer loop selects exercises and guides the student through te curriculum \citep{vanlehn2006}.
The student model helps to make these loops adaptive to student needs.
In a later article loops were described as  feedback loops \citep{loops}.

Developing a student model has software engineering dimensions, such as  requirements analysis and architecture \citep{Hatzilygeroudis2004, Jeremi2004}.
Any student model starts with log data. 
The PSLC DataShop collects logdata from \glspl{its} in many domains   \citep{VanLehn2007}.
This requires attention to qualities such as security, privacy and standardisation of dataformats.



\subsection{Educational theories for \gls{its} design}
Theories of learning play an important role in the development of any \gls{its},
and vice versa \glspl{its} have also inspired the forming of learning theories. 
The Lisp tutor was based on the \gls{actr} theory of cognition \citep{anderson_1996}. 
Problem solving tasks can be carried out by executing a set of rules.
The human mind must contain large numbers of rules, some of which are obtained in exercises with an ITS.
Novices learn rules and apply them step by step, whereas experts apply rules automatically \citep{skill}.
The exercises in an \gls{its} are designed to automate rule application in the mind.

In this thesis we follow the \gls{tensteps} for instructional design.
This is a methodology for learning complex skills. 
The methodology is based on learning theories such as \gls{actr}.


\subsection{IDEAS domain reasoners in Functional Programming}

The IDEAS project has developed domain reasoners that can track the steps in a solution strategy in an exercise \citep{ideas1, UUCS2014005}.
These domain reasoners can be applied in many domains; logic, geometry, microprocessor programming or imperative programming \citep{p43-keuning}.
In this paper we concentrate on domain reasoners in the area of functional programming in Haskell.
Many universities teach Haskell to students to become acquainted with functional programming concepts \citep{hutton_2016}.

The Haskell Expression Evaluator (\gls{hee}) is a tutor with which the student can stepwise evaluate Haskell expressions \citep{tutorHEE}. 
Understanding expression evaluation is essential for the understanding of functional programming.
Students with a background in imperative program have to make a mind shift when starting with Haskell.

The \gls{askelle} tutor can conduct exercises where the student has to create a small function in Haskell \citep{ideas2}.
\Gls{askelle} can conduct class 3 exercises \citep{exerciseClasses}, where there are multiple solutions strategies \citep{Gerdes2017}.


\section{Our contribution: lightweight student modelling.}

\subsection{Problem statement and research question}
\label{sec:problstmt}
The above mentioned IDEAS tutors are stateless. 
The tutor does not keep track of which exercises the student has completed.
The outer loop of Ask-Elle is a menu of exercises. 

As a student progresses through the curriculum we must recommend exercises that are not too difficult and not too easy.
The outer loop of an \gls{its}   recommends exercises and  should adapt to student needs and preferences. 
A fast student may skip exercises that are too easy, the system must not insist on completing easier exercises even when they are prerequisite.
A slow student may need  more exercises.

Information on student progress is kept in a student model.
The student model can be queried during exercises, to choose between interventions. 
Typical interventions are: reminding, persuasion, teaching and remediation \citep{loops}.

Exercises and model solutions must be linked to domain concepts and learning objectives.
The model must make it easy for the author to specify exercises.

The student model must be updated after each step. 
Some inferencing mechanism is needed to infer the students cognitive state from the task step history.
Machine learning and Bayesian models are often used as intelligent technology for knowledge inferencing.
These technologies require a lot of data and effort to set up and parameterise.
We are looking for a lightweight solution for knowledge representation and knowledge inferencing.
This must be easy to implement in research and development situations where resources are limited.

The central question in this research is how we can build a lightweight student model that supports the \gls{tensteps} and meets the requirements mentioned.

\subsection{Our solution}
An overlay model is a model that overlays a domain model, by assigning a value to each domain concept.
Overlay models are often used as the first step for introduction of a student model.
Most \glspl{its} use an overlay model in their student model, often combined with other approaches \citep{chrysafiadi_2013}.

We use lattice theory to develop an overlay model and a rule system for knowledge tracing. 
The ruleset follows the \gls{tensteps}, but is not specific for any domain.
The author only has to specify properties for exercises.

We show how the problems mentioned above are solved in some scenarios from a functional programming curriculum.
We found that the scenarios specified could be implemented with a very limited ruleset.
With this approach a student model can be introduced in a new \gls{its} with limited effort.



\subsection{Thesis organisation}
This thesis is organised as follows. 
Chapter 2 discusses the components that provide the foundation for our work.
We combine concepts from the \gls{tensteps}, mathematics (lattice theory), domain reasoners and \gls{its} and functional programming.
This chapter provides a brief introduction to these concepts.
Our method for student modelling and inferencing is explained in chapter 3.
In chapter 4 we demonstrate how this works for some scenarios.
In chapter 5 we analyse the threats to validity, the conclusions and future work.

















