\chapter{Conclusions}

In this experiment we have build software that records a trace of student interactions.
We have shown how the reasoning can take place with the help of lattices and progress vectors.

\section{Research problem}\label{sec:questions}

The objective to this research is to find answers to the following question:

\begin{center}
\textbf{\\How can a student model based on the \gls{tensteps} methodology be created and used in Ask-Elle for inner and outer loop behavior ?}
\end{center}

This question can be divided in the following subquestions.

\begin{enumerate}[Q1.]
\item How can a student model be used to diagnose learning problems such as missing prerequisites and forgotten or misunderstood topics?\\


\item How can the student model be used to choose between possible interventions?\\
Typical interventions are: reminding, persuasion, teaching and remediation \citep{loops}.

\item How can the inner loop behavior be used for knowledge tracing in the student model?\\
Completing exercises provides evidence of student mastery.
In the \gls{tensteps} methodology the amount of guidance needed is an important indicator of student mastery.
If different solutions are possible then each solution may provide evidence for mastery of different concepts.

\item How can the outer loop control the scheduling of learning tasks and task classes following the \gls{tensteps} methodology?\\
In the \gls{tensteps} the student can progress to the next task class when no procedural guidance is needed anymore.
Variability of practice is important in the \gls{tensteps} methodology for a good transfer of learning.

\item What is the impact of the student model on content authoring?\\
Exercises and model solutions must be linked to domain concepts and learning objectives.
This makes the authoring of artifacts more complex. 

\item How can we build a lightweight knowledge representation and knowledge inference mechanism?\\
Our objective is to find a lightweight approach that is adequate for the \gls{tensteps} approach.
In the \gls{tensteps} knowledge engineering is positioned as an instructional design aspect in designing supportive information.
Lightweight means that it is easy to extract the knowledge ontology from the course materials such as exercises and books.

Some inference mechanism is needed to infer student knowledge from the interactions of the student with the tutor system.
Many ITS's use algorithms such as Bayesian logic, fuzzy logic and machine learning \citep{chrysafiadi_2013}.
These algorithms are heavyweight in the sense that they need specialized fitting of parameters, lots of data, and have a high computational complexity.
The \gls{tensteps} has simple rules for task class promotion.
It must be possible to find a lightweight inference mechanism without parameter fitting and computational problems to achieve this reasoning.

\end{enumerate}


\section{Research questions}
So how does this answer the questions we posed in section \ref{sec:questions}?

\begin{enumerate}[Q1.]
\item How can a student model be used to diagnose learning problems such as missing prerequisites and forgotten or misunderstood topics?\\
We have shown that we can test for missing prerequisites.
It is possible to define a rule that display the missing prerequisites at the beginning of an exercise.
We can define prequisites on the level of knowledge and point the student to relevant supportive information.

\item How can the student model be used to choose between possible interventions?\\
To be further investigated.

\item How can the inner loop behavior be used for knowledge tracing in the student model?\\
To be further investigated.

\item How can the outer loop control the scheduling of learning tasks and task classes according the Ten Steps?\\
We have shown that with careful definition of prerequisites and objectives the student guided through the exercises.
We have structured the ontology model according to the TenSteps. 
This allows the implementation of the relevant principles of the TenSteps.

\item What is the impact of the student model on content authoring?\\
Simple things are easy to implement. 
By carefully defining prerequisites objectives and completion criteria it is possible to guide the student
through the curriculum in a very adaptive way.
It is not necessary to store this centrally in the model. 
These components can be stored with the exercises in a domain reasoner and to retrieve them with additional services.

\item How can we build a lightweight knowledge representation and knowledge inference mechanism?\\
As we have shown, the mathematical lattice theory and partial orderings ar a natural fit to describe student progress.
By defining exercise and task class objectives and prerequisites as destinations we can guide 
the student on his route to an end-goals in a similar way as a car navigation system guides the driver to his destination.
We just have more then two dimensions.

\end{enumerate}

\section{Threats to validity}
\begin{itemize}
\item Other use cases might present challenges to transformation of domain reasoner responses to step vector.
There may be situations where counting rules is not appropriate.

\item A central principle in the TenSteps is whole task support in exercises.
The material that we considered here is only a prelude for exercises of more complexity.
The TenSteps recognises that it is sometimes necessary to learn constituent skills before a complete task can be attempted.
A significant part of the TenSteps can be applied in this situation also.


\item The theories have only been tested against fictitious scenarios, not with real students.
\item The exercises are executed with simplified domain reasoners, also specifically designed for this experiment.
\item We have assumed that domain reasoners be adapted to interface with our ontology.
There is no adapted domain reasoner, and we have not run it end to end.
\end{itemize}



\section{Future work}

We hope this experiment will encourage other researcher to explore domain reasoners, the TenSteps and lattice theory as a basis for \gls{its} design.

As this is a first attempt to create an outer loop for domain reasoners a lot of questions remain.

\begin{itemize}

\item We have only carried out very limited experiments to run our inference mechanism with other existing domain reasoners.
\item Our engine may be sensitive to rule orderings and overlapping conditions.
To ease the reasoning about this we have required monotone growth of the model.
How can we formally find and test under which conditions a fixpoint will be reached?
\item How can the system be formally model tested and verified.
It is possible to define constraints on the ontology.
In model testing constraints are defined for safety, fairness and liveliness.
If we start a task class, then there must always be a path complete it successfully.
We can ask a domain reasoner for a set of model solutions for an exercises. 
In that case each solution must successfully complete the exercise.
\item Is it possible to integrate our lattice datatype with existing inferencing engines such as Pellet or Drools.
\item Can this lattice theory of progress vectors be of use in other business rule engines in other domains then \gls{its}.
\end{itemize}




