\chapter{Validation}
\label{ch:valid}
The purpose of this design is find a way to build a lightweight student model, that can be used with IDEAS domain reasoners.
We use the \gls{tensteps} as a instructional framework.
Section \ref{tsblueprint} described the blueprint four components.
We use the blueprint to structure our educational model  (section \ref{pv4sm}).
The model is updated with  step vector. 
After this update  rules are executed to derive now facts, that are joined into the model.

In this chapter we present some scenarios that illustrate how this system can be used.
We have created two examples of typical domain reasoners where information from the diagnose service can be used.
Section \ref{progtut} presents some arguments why a programming tutor differs from these cases.

\section{Outer loop scenarios}


\subsubsection{Exercises for the outer loop test}

The scenarios are based on the exercises with double maxi and combined, that we described in section \ref{sec:drmaxidble}.

\subsection{Educational model for task class on expression evaluation}

We have to define a progress vector for the task class, with a progress vector per exercise.
The progress vector for an exercise in this task class contain subvectors:

\begin{table}[H]
\begin{tabular}{| l | l |}
\hline
Exercise subvector & Progress indicator name\\
\hline
Prerequisites& ES.prerequisite\\
Objectives& ES.objective\\
Succes criteria& ES.successCriteria\\
Progress information& DR.Rules\\
\hline
\end{tabular}
\caption{ES.prerequisite}
\label{double.obj}
\end{table}

The state of an exercise is kept in a set of boolean values.
Below we show the possible content of the vectors.


\subsubsection{Objectives}
\begin{table}[H]
\begin{tabular}{| l | l | l |}
\hline
Key & Progress indicator value& exercise type\\
\hline
EX.double& True & double\\
EX.maxi& True & maxi\\
EX.combined& True& combined\\
CAUS.expressionEvaluation& Comprehension& combined\\
PRE.plus& Comprehension& double, combined\\
PRE.gtr& Comprehension& maxi, combined\\
SYN.if & Comprehension& maxi, combined\\
\hline
\end{tabular}
\caption{Possible values in subvector ES.objective}
\label{double.obj}
\end{table}

For each type of exercise we keep track if the student mastery.
Successful completion of an exercise of type double is recorded with the indicator EX.double for example.

We have adopted a naming convention for domain concepts with prefixes.
The prefix CAUS is causal principle, SYN a syntax construct, and PRE the prelude.
Depending on the exercise type we collect evidence for domain concepts.

A subset of these values appears in the objectives of each exercise.
On successful exercise completion these values are joined to the student model.
The model is stored in subvector ES.student of the outer model.
The student model contains domain concepts, and exercise mastery.





\subsubsection{Prerequisites}
\begin{table}[H]
\begin{tabular}{| l | l | l |}
\hline
Key & Progress indicator value& exercisetype\\
\hline
EX.double& True& maxi\\
EX.maxi& True & combined\\
\hline
\end{tabular}
\caption{Possible values in subvector ES.prerequisite}
\label{double.obj}
\end{table}

Exercises of type double are at the beginning of the curriculum and have no prerequisites.
An exercise of type maxi comes after double, thus the vector contains EX.double.
The student meets the prerequisites if $\mathit{ES.student \geq  ES.prerequisite}$.

There are two styles of specifying prerequisites, in terms of previous exercises (or task classes) or in terms of required skills.
We also might specify PRE.plus as prerequisite for exercise combined.

It is a matter of user interface design how this is used.
It is not necessary to prevent students from taking the exercise, it can also be a recommendation.

\subsubsection{Step vector}
\begin{table}[H]
\begin{tabular}{| l | l | l |}
\hline
Key & Value& Meaning\\
\hline
DR.rules/RL.expr.double& 1&  Expected: rule\\
DR.rules/RL.expr.plus& 1& Expected: rule\\
DR.rules/RL.expr.gtr& 1& Expected: rule\\
DR.rules/RL.expr.if& 1& Expected: rule\\
DR.rules/RL.expr.maxi& 1& Expected: rule\\
DR.rules/ES.error & True & Error e.g. NotEquivalent\\
DR.created & True & the exercise is started\\
ES.finished & True& Diagnose: Expected + finished\\
\hline
\end{tabular}
\caption{Possible values in a step vector}
\label{double.step}
\end{table}

When a step is executed a step vector is inferred from the domain reasoner response.
Other educational components may als submit step vectors, e.q. when the student reads a page.
The step vector contains a progress vector with the same position in the model as the exercise.

We have adopted the naming convention of prefixing rule with ``RL.''.
The step vector is added to the model (with $\addjoin$), thus we count rule executions in a subvector DR.rules.
When the exercise is created or finished some flags are set directly in the exercise vector.
When an error is made the indicator is placed in DR.rules and joined to the exercise level.

This is just an example of a transformation from a diagnose to step vector.
Depending on rule design other transformations are possible, or necessary.
For example the Haskell Expression Evaluator derives rules for each element in the prelude.
So from specific rule executions we may derive student knowledge, which might be positioned elsewhere in the model.
Theoretically a step vector may even update the student model directly.

\subsubsection{Success criteria}
\begin{table}[H]
\begin{tabular}{| l | l |}
\hline
Key & Progress indicator value\\
\hline
RL.expr.double& n\\
RL.expr.plus& n\\
RL.expr.gtr& n\\
RL.expr.if& n\\
RL.expr.maxi& n\\
\hline
\end{tabular}
\caption{Success criteria (ES.successCriteria)}
\label{double.success}
\end{table}

A condition for successful completion of an exercise is $\mathit{DR.rules \geq ES.successCriteria}$.
We have exercises with double nested twice. 
The student could just enter the final answer. 
That is probably correct, but provides no evidence that he understands the individual steps.
With this mechanism we make sure that rules expr.double and expr.plus are executed twice before marking the exercise successful.

Some exercises have multiple solution path. 
In that case it may be good idea to validate that each solution path evaluates to success.
A normal student may need one demo, get it and complete the next exercise.
A slow student may make more mistakes, and will need more attempts.




\subsection{Guiding the student through the curriculum}
A slow student may need more exercises or support to complete the task class.
We stated earlier that we wanted the student to complete the exercise double and then maxi followed by combined.
This is achieved by specifying prerequisites for exercises maxi and combined.

\begin{table}[H]
\begin{tabular}{| l | l |}
\hline
Key & Progress indicator value\\
\hline
CS.double& True\\
\hline
\end{tabular}
\caption{Prerequisites for exercise maxi}
\label{maxi.preq}
\end{table}

\begin{table}[H]
\begin{tabular}{| l | l |}
\hline
Key & Progress indicator value\\
\hline
CS.maxi& True\\
\hline
\end{tabular}
\caption{Objectives for exercise maxi}
\label{maxi.obj}
\end{table}

Tables \ref{maxi.preq} and \ref{maxi.obj} show the objectives and prerequisites of the slightly more advanced exercise maxi.

What is done with prerequisites is a matter of user interface design.
We do not assume that the student is inhibited from executing exercises for which there is no evidence in the student model.
The user interface may query the student model at any time.
A logical choice would be to recommend exercises for  for which the student meets the
prerequisites, but has not yet reached the objectives. 


?????????


If the student does not yet feel comfortable, he can always take some extra exercises.
Off course only exercises in state INIT can be put in a menu.
Only one exercise must be in state BUSY, and exercises in the final states can be shown in some progress report.

In this example, the student would get exercises maxi recommended after double is completed. 
When maxi is completed, the next challenge is to attempt a combined exercise.

VanLehn mentions that it is not always desirable to make the \gls{its} adaptive to student speed \citep{vanlehn2006}.
In a class situation it is desirable to keep the class together.
So the challenge for the instructional designer is to decide how to be adaptive.
For example for faster students some optional more challenging exercises may be added.

\subsection{Fast students and Knowledge space rule.}

Suppose a student just starts with the last and most difficult exercise and completes it successfully.
In that case it make no sense to insist on completing the other exercises that are much easier.
We are going to update the student model backward to mark the objectives and prerequisites met.
This works with the fast student rule (section \ref{sec:fststdntrule}).

The prerequisite of combined ($\mathit{EX.maxi \; True}$) is added to $\mathit{ES.student}$.
A second run of the rule will then fire on exercise type maxi and add ($\mathit{EX.double \; True}$).
Then a fixpoint is reached, with all exercise types completed.

The conclusion is that with this rule a fast student can be allowed to start anywhere in the curriculum.
If he is successful, the system will not recommend exercises that are too easy. 
This can make the \gls{its} adaptive to the entry level of students.

\section{Inner loop scenarios.}

For the inner loop scenarios we have created a simple domain reasoner for solving simple linear equations with integer coefficients.
We have chosen this example because it allows us to demonstrate cases with a misconception, prerequisite problems and multiple solution strategies.

\subsection{An exercise with misconceptions and variations}
Linear equations can be solved by eliminating the $x$ variable. 
In figure \ref{fig:deriv} derivation \#1 shows a printout of a typical correct solution.
For each step the domain reasoner prints the name of rule that was applied.
The students must determine the least common multiple (lcm) of the $x$-coefficents.
Next the student must multiply the equations to make the $x$-coefficents equal to the lcm.
One equation must be subtracted from the other, so that $y$ can be computed. 
This can be substituted and then $x$ computed.

\begin{figure}
\begin{minipage}[t]{7cm}
\begin{verbatim}
Derivation #1
6x + y = 22
4x + 2y = 20
   => eval.x.factors.equal
12x + 2y = 44
12x + 6y = 60
   => eval.subtract.e1.from.e2
12x + 2y = 44
4y = 16
   => eval.compute.y
12x + 2y = 44
y = 4
   => eval.substitute.y
12x + 8 = 44
y = 4
   => eval.move.constant.right
12x = 36
y = 4
   => eval.compute.x
x = 3
y = 4

\end{verbatim}
\end{minipage}
\begin{minipage}[t]{3cm}
\begin{verbatim}
Derivation #2
6x + y = 22
4x + 2y = 20
   => eval.y.factors.equal
12x + 2y = 44
4x + 2y = 20
   => eval.subtract.e1.from.e2
12x + 2y = 44
-8x = -24
   => eval.compute.x
12x + 2y = 44
x = 3
...
Derivation #3
...
   => eval.substitute.y
12x + 8 = 44
y = 4
   => buggy.move
12x = 52
y = 4
   => eval.compute.x
x = 4
y = 4

\end{verbatim}
\end{minipage}
\caption{Derivations}\label{fig:deriv}
\end{figure}

We will use this example  in three cases.

\begin{description}
\item[Variations]
Derivation \#2 shows a variation where the $y$ is eliminated, and $x$ computed first. 
The \gls{tensteps} advices to present multiple variations.
We want the student to encounter both variations.
\item[Misconception]
Derivation \#3 show a misconception.
After the y is substituted the student has to solve $ 12 x + 8 = 44$. 
Student have to subtract $8$ from both sides of the equation.
People sometimes talk about moving the $8$ to the other side.
Some students misunderstand this and add the $8$ to get $52$.
The domain reasoner recognises this, and a buggy rule (buggy.move) is reported back.
 \item[Missing prerequisite]
A prerequisite is that the student knows what a lcm is, and how to compute this.
We may or may not have information about this in the student model, and this may or may not be accurate.
We must see how to react in these cases.

\end{description}


\subsection{Reasoning about variations}
\label{sec:variat}

The system has a variation in that we may eliminate either $x$ or $y$.
In both cases a different set of rules is applied.
\begin{table}[H]
\begin{tabular}{| l | l | l |}
\hline
Key & PI value& Variant\\
\hline
RL.eval.factors.equal& True& variant xy\\
ES.objective/VARIANT.x& True& variant xy\\
ES.objective/VARIANT.y& True& variant xy\\
\hline
RL.eval.x.factors.equal&True& variant x\\
\hline
RL.eval.y.factors.equal&True& variant y\\
\hline
\end{tabular}
\caption{Step vector values for first rule}
\label{step.1.step}
\end{table}

\begin{table}[H]
\begin{tabular}{| l | l | l |}
\hline
Key & PI value& Variant\\
\hline
ES.objective/VARIANT.xy& True& variant xy\\
\hline
ES.objective/VARIANT.x& True& variant x\\
\hline
ES.objective/VARIANT.y& True& variant y\\
\hline
\end{tabular}
\caption{Objectives}
\label{step.1.obj}
\end{table}


One solution is to have three types of exercises, variant x, y and xy.
The student starts with exercise of variant xy and can take any solution strategy.
For exercises variantXY we must calculate the step vector as in the upper part of table \ref{step.1.step}.
From the rulename we must leave out the $x$ or $y$.
The variant must be calculated as well and entered in the objectives.
So the student achieves two objectives VARIANT.xy and one of VARIANT.x or VARIANT.y.

The specific exercise for variant x has prerequisite VARIANT.xy and objective VARIANT.x, and similar for variant y.
Let us assume the student took variant x.
Then the system will now recommend variant y, since variant x is already achieved.
This works in a similar to the way we guided the student along exercises double, maxi and combined.

For exercises of variant y we leave the y in the step vector.
The student is now required to take this specific variant, so the other rule is not permitted.
If the student eliminates the $x$ then the system must give feedback.
The user interface can inspect the student model to detect this when $\mathit{ES.successCriteria}$ and the $\mathit{DR.Rules}$ are not comparable.


\subsection{Reasoning about prerequisites}
\label{sec:preqsts}

\begin{table}[H]
\begin{tabular}{| l | l | l |}
\hline
Key & PI value\\
\hline
ES.prerequisite/MATH.leastCommonMultiple & Knowledge\\
\hline
\end{tabular}
\caption{Prerequisite}
\label{preq.step}
\end{table}

\subsubsection{Prerequisites missing}
When the student starts the exercise and we do not have the evidence in the student model, an intervention can be triggered.
The system may point this out the the student, and offer some part-task practice, if that was not allready recommended automatically. 
Alternatively, the student may assert that he has the knowledge. 
In both cases the student model will be updated to reflect this.
This method can also be used to point the student to specific resources in supportive materials.

\subsubsection{Evidence present}

The condition may detected by a buggy rule firing.
In that case the system may give immediate feedback.
This is likely a slip, the exercise may be restarted.

\subsection{Reasoning with misconceptions}

The student moves a constant to the right by adding it instead of subtracting it.
This is a misconception, as the equations are not equivalent.
We can make a part-task practice with exercises on this aspect.

Part-task practices can be recommended in a similar way as exercises, if the prerequisites are met, yet the objectives not achieved.
Except in this case the prerequisite is that the student has a misconception.


\begin{table}[H]
\begin{tabular}{| l | l | l |}
\hline
Key & PI value\\
\hline
DR.Buggy/BUG.wrongMove & 1\\
\hline
\end{tabular}
\caption{Step vector for buggy rule}
\label{buggy.step}
\end{table}


In table \ref{buggy.step} the step vector is outlined. 
We better collect buggy rules separate, and not mix them with the DR.Rules.
The rule fires if there are any errors when the exercise is finished.
In fact  we must add the rule to the other rules explained in section \ref{subs:rules}.
This means that a predicate $\mathit{ /\$exercise/DR.Buggy == bottom}$ must be added to the other three rules to make the rules mutually exclusive.
Another option would be add the DR.Buggy action to the existing error rule and set the $\mathit{/\$exercise/DR.rules/ES.error}$ in the step vector.

The name of the buggy is used in the student model to indicate the student suffered from the misconception.
A skill FIX.wrongMove can be set as objective for the part-task practice.
Upon completion of the part-task practice the student model will be updated and the part-task is no longer recommended.
Part-task practices state can be inferred  with the same or similar rules as exercises.

The user interface should provide feedback immediately, and it is best to terminate the exercise immediately.
This has to deal with the situation that the error was fixed before, in that case it is a slip and a reminder should be enough.


\section{Programming tutors}
\label{progtut}

Certain educational problems lend themselves better for implementation in an \gls{its} then others.
Well defined problems are forexample  those in mathematics physics and computer language.
These problems are easier to implement in an \gls{its} than for example swimming, drawing, fashion design or presentation skills.



Le, Loll and Pinkwart have defined a classification scheme for  educational problems \cite{exerciseClasses}.
Educational problems are classified in five classes:
\begin{enumerate}
\item problems with one single solution
\item problems with one solution strategy which can be implemented with different variants
\item problems with a  limited number of typical solution strategies
\item problems with a great variety of solution strategies
\item problems where solutions can not be verified automatically.
\end{enumerate}

The expression evaluation exercises may be classified as class one. 
Different solution path's are possible.
In this exercise we do not care much which path the student takes.

The linear equations exercise has multiple variants of the solution, and we do care much about which variant is taken.
So this is a class two exercise.
Step five in the \gls{tensteps} is about cognitive strategies.
Part of this step is finding a systematic approach to problem solving (\gls{sap}).
The linear equation example has a well defined \gls{sap}, from which we designed the rules.

Programming tutors such as \gls{askelle} are class 3.
They do not start with a predefined strategy, but with model solutions for which a strategy is derived.
The set of solutions is not exhaustive. 
The student may arrive at a completely different solution, that is nevertheless completely valid.

A domain reasoner like \gls{askelle} functions like a set of training wheels.
If needed the tutor can give you a hint or a demonstration.
Ultimately we want the student to execute the exercise without this procedural support.
Using the techniques show before we can guide the student to exercises for which hints are turned off.

A problem is how to update the student model.
We can not count on inferences from  rule executions, as the tutor may not have recognised these.

For this class of tutors we propose a constraint based update scheme.
Domain reasoners parse the solution, and they can provide a step vector with information on solution conditions.
This may include information on solutions correctness, syntax, prelude and language concepts used.
For example the solutions in listing  \ref{sol2} might provide information on domain concepts.

\begin{table}[H]
\begin{tabular}{| l | l | l |}
\hline
Domain concept & solution nr\\
\hline
prelude: length &  1\\
\hline
syntax: if &  2\\
prelude: null &  2\\
\hline
syntax: pattern matching on lists &  3\\
syntax: where &  3\\
concept: strict evaluation & 3\\
concept: recursion &  3\\
\hline
concept: higher order function  &  4\\
prelude: foldl &  4\\
\hline
\end{tabular}
\caption{Domain concepts in different solutions}
\label{table.cls3}
\end{table}

Table \ref{table.cls3} shows which concepts from functional programming, syntax and prelude were used in different
solutions for the same problem.
These concepts can be put in step vectors and used to update the student model, when the exercise is completed successfully.

The updated student model can be used in a task class to steer the student towards solutions.
In a side bar we can show a list of domain concepts that we want him to apply.
During the course of exercises the concepts used are removed, and the student must clean the bar. 
If another solution variant is available for an exercise we may refuse a solution that only has familiar concepts.
Some creativity from the instructional designer helps to integrate the student model in user experience design.














