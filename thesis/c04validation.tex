\chapter{Validation}

In section \ref{sec:problstmt} we raised the question how to build a lightweight student model that can be used with IDEAS domain reasoners.
We use the \gls{tensteps} as instructional framework.
Section \ref{tsblueprint} described the four components of the training blueprint.
We use the blueprint to structure our educational model  (Section \ref{pv4sm}).

A student model is derived from a log of the student interactions \citep{studentModelling}.
We translate the student interaction in a sequence of step vectors for each task step.
We update the model by applying the step vectors in the correct sequence.
After each update,  the rules are executed to derive new facts, which are joined into the model.

In this chapter we present some scenarios that illustrate how this system can be used.
We have created two examples of typical domain reasoners where information from the diagnose service can be used.
Section \ref{progtut} presents some arguments why a programming tutor differs from these cases.

\section{Choice of scenarios}
An important characteristic of \glspl{its} is adaptability.
A pilot is trained to land an airplane safely in any condition, and it does not matter if it took six or ten lessons.
A violin teacher on the other hand may try to get the most out of a talented student.

Curriculum designers must make conscious decisions on adaptability.
For example, in a classroom situation it is not practical to have students work on different subjects depending on how fast the are \citep{vanlehn2006}.

The \gls{tensteps} methodology aims at achieving a minimum level of mastery in complex skills.
Our student model therefore must adapt to the start situation.
Most students complete the curriculum in an average number of lessons.
Some students are slower and need some more exercises.
One student was sick, missed the previous module and does not meet the prerequisites.
Another student took the module but forgot some important details.
The next student had to repeat the class, but he already masters this topic.
Students may not read carefully and acquire a misconception.

The student model should be able to deal with these problems.
The student model makes the \gls{its}  a regulating feedback control system similar to a thermostat.
The scenarios verify that our solution has the potential to meet the requirements.
Indeed we abandoned some other attempt because that approach led to more complexity.

The scenario on variations is an exeception, it is not related to the input level.
Variability is  a principle from the \gls{tensteps}.



\section{Scenario: A normal student}

The first exercise is to evaluate the expression $\mathit{double ( double ( double \;  5 ))}$.




\begin{table}[H]
\begin{tabular}{| l | l | l | l | l | l |}
\hline
Step & Component & indicator & indicator & value& expression entered\\
\hline
0 & LT.01.double  & DR.created & True & & $\mathit{double ( double ( double  5 ))}$\\
\hline
1& LT.01.double & DR.Rules & RL.expr.double & 1 & $\mathit{double ( double  ( 5  +  5 ))}$ \\
\hline
2&LT.01.double & DR.Rules  & RL.expr.plus & 1 & $\mathit{double ( double \; 10)}$ \\
\hline
3& LT.01.double & DR.Rules & RL.expr.double & 1 & $\mathit{double (10  +  10))}$\\
\hline
4& LT.01.double & DR.Rules  & RL.expr.plus & 1 & $\mathit{double \;  20}$\\
\hline
5 &LT.01.double & DR.Rules & RL.expr.double & 1 & $\mathit{20  +  20}$ \\
\hline
6& LT.01.double & DR.Rules  & RL.expr.plus & 1 & $\mathit{40}$\\
& & ES.finished & True & &\\
\hline
\end{tabular}
\caption{Step vectors}
\label{double.steps}
\end{table}

Table \ref{double.steps} shows the step vectors for steps taken.
The vector for step 0 was inferred from a  message that started the exercise.
The indicator DR.created is one of the flags from which the exercise state is inferred, and the exercise now in state BUSY.
This was a triple nested double exercise. The last vector contains the finished flag that triggers the rule.


\begin{table}[H]
\begin{tabular}{| l | l | l | l |}
\hline
Component & indicator & indicator & value\\
\hline
LT.01.double & ES.objective &EX.double & True\\
 & & PRE.plus &Comprehension\\
 & ES.successCriteria &RL.expr.double & 1\\
 & & RL.expr.plus &1\\
\hline
\end{tabular}
\caption{Exercise double at start of exercise}
\label{double.before}
\end{table}

Table \ref{double.before} shows the content of the exercise part of the model at the start of an exercises.







\begin{table}[H]
\begin{tabular}{| l | l | l | l |}
\hline
Component & indicator & indicator & value\\
\hline
LT.01.double & ES.objective &EX.double & True\\
 & & PRE.plus &Comprehension\\
 & ES.successCriteria &RL.expr.double & 1\\
 & & RL.expr.plus &1\\
& DR.Rules & RL.expr.double & 3 \\
& & RL.expr.plus & 3\\
& ES.finished & True & \\
& DR.created & True & \\
\hline
\end{tabular}
\caption{Step vectors added to exercise}
\label{double.stepsadded}
\end{table}

Table \ref{double.stepsadded} show the exercise after all step vectors have been added.
The last vector has set the finished flag. 
The engine will now fire the Succes rule.

\begin{table}[H]
\begin{tabular}{| l | l | l | l |}
\hline
Component & indicator & indicator & value\\
\hline
LT.01.double & ES.objective &EX.double & True\\
 & & PRE.plus &Comprehension\\
 & ES.successCriteria &RL.expr.double & 1\\
 & & RL.expr.plus &1\\
& DR.Rules & RL.expr.double & 3 \\
& & RL.expr.plus & 3\\
& ES.finished & True & \\
& DR.created & True & \\
& ES.success & True &\\
\hline
\end{tabular}
\caption{State after rule execution of exercises double}
\label{double.final}
\end{table}

\begin{table}[H]
\begin{tabular}{| l | l | l | l}
\hline
Component & indicator & value\\
\hline
ES.student & EX.double & True \\
& PRE.plus & Comprehension  \\
\hline
\end{tabular}

\caption{Updated student model after exercise double}
\label{double.student}
\end{table}

Table \ref{double.final} and table \ref{double.student} show the outcome of the rule execution.
The exercise has the success flag set. 
It may be displayed as such in the user interface, and it prevents the rules from firing again.
The student model was empty and now the objectives have been added.
The student has shown that he comprehends the prelude function plus.
The student model also indicates that he has executed an exercise of type double.

\section{Scenario: A missing prerequisite}
 \label{sec:preqsts}
\begin{table}[H]
\begin{tabular}{| l | l | l | l |}
\hline
Component & indicator & indicator & value\\
\hline
LT.11.maxi & ES.prerequisite &EX.double & True\\
&ES.objective & EX.maxi & True\\
& & PRE.gtr &  Comprehension\\
& & SYN.if & Comprehension\\
& ES.successCriteria & RL.expr.gtr & ..\\
\hline
\end{tabular}
\caption{Prerequisites for exercise maxi}
\label{maxi.preq}
\end{table}

Table \ref{maxi.preq} shows  part of the content of an exercise of type maxi.
Note that maxi has a prerequisite of EX.double, which the normal student meets after exercise double.
The predicates in the exercise recommendation query (section \ref{sec:exrecquer}) can be used to determine if the
exercise is difficult (prerequisites not met),  passed (objectives already met) or the right level.
Exercises of type double is now classified as passed, of course the student can take more exercises double if desired.
We now recommend exercise maxi, the student meets the prerequisites, but has not yet met the objectives.
An exercise of type combined has EX.maxi  as prerequisite and is classified as difficult.
This way we can guide the student through the curriculum.

If the student would start an exercise without meeting this prerequisite, the user interface
may check this and display a message recommending a resource to fill the gap.


\section{Scenario: A slow student}

 \label{sec:slowStudent}
\begin{table}[H]
\begin{tabular}{| l | l | l | l |}
\hline
Component & indicator & indicator & value\\
\hline
LT.11.maxi  & DR.Rules & RL.expr.maxi & 1\\
& DR.created & True & \\
& ES.failure &True & \\
& ES.finished &True & \\
&ES.objective & EX.maxi & True\\
& & PRE.gtr &  Comprehension\\
& & SYN.if & Comprehension\\
& ES.prerequisite &EX.double & True\\

& ES.successCriteria & RL.expr.gtr & 1\\
& & RL.expr.if & 1 \\
& & RL.expr.maxi & 1\\

\hline
\end{tabular}
\caption{Indicators for exercise maxi after error}
\label{maxi.error.ex}
\end{table}


\begin{table}[H]
\begin{tabular}{| l | l | l | l}
\hline
Component & indicator & value\\
\hline
ES.student & EX.double & True \\
& PRE.plus & Comprehension  \\
\hline
\end{tabular}
\caption{Updated student model after exercise maxi}
\label{maxi.error.student}
\end{table}

Table \ref{maxi.error.ex} shows the state of the exercise after the exercise was finished and the rules executed.
The exercise was finished, but in the DR.Rules vector we see that only rule expr.maxi was executed once.
This is not good enough, we have not seen expr.if and expr.gtr.
The success criteria are not met and the exercise enters state FAILURE indicated by ES.failure true.

Table \ref{maxi.error.student} show the student model.
The student completed the double exercise, but it was not updated with the objectives of maxi.

We can not infer what exactly happened.
The student may have requested procedural support, and the domain reasoner demonstrated the steps.
Another possibility is that the student may have just typed the final answer and skipped steps.
A hint may be accounted for by setting a low PiBloom indicator (Knowledge) on a topic.
We did not make this refinement here.





\section{Scenario: A fast student}
\label{sec:fastStudent}
\begin{table}[H]
\begin{tabular}{| l | l | l | l |}
\hline
Component & indicator & indicator & value\\
\hline
LT22.combined & DR.Rules &RL.expr.double &  1 \\
 & & RL.expr.gtr  & 1\\
 & &  RL.expr.if & 1\\
 & & RL.expr.maxi  & 1\\
 & & RL.expr.plus  & 1\\
 & DR.created &  True & \\
 & ES.finished &  True & \\
 & ES.objective &  CAUS.expressionEvaluation & Comprehension \\
 & &  EX.combined & True \\
 & &  PRE.gtr & Comprehension\\
 & &  PRE.plus & Comprehension\\
 & &  SYN.if & Comprehension \\
 & ES.prerequisite& EX.maxi& True  \\
 & ES.success & True & \\
 & ES.successCriteria & RL.expr.double& 1  \\
 & & RL.expr.gtr &  1 \\
 & & RL.expr.if &  1 \\
 & & RL.expr.maxi&  1 \\
 & & RL.expr.plus &  1 \\
\hline
\end{tabular}
\caption{Indicators for exercise combined for a fast student}
\label{combined.fast.ex}
\end{table}


\begin{table}[H]
\begin{tabular}{| l | l | l | l}
\hline
Component & indicator & value\\
\hline
ES.student & CAUS.expressionEvaluation & Comprehension\\
 &  EX.combined&  True \\
 &  EX.double & True  \\
 & EX.maxi &  True \\
 & PRE.gtr &  Comprehension \\
 & PRE.plus & Comprehension  \\
 & SYN.if &  Comprehension \\
\hline
\end{tabular}
\caption{Updated student model after exercise combined}
\label{combined.fast.student}
\end{table}

This student started immediately with exercise combined, ignoring any messages on missing prerequisites.
Table \ref{combined.fast.ex} shows the educational state of the exercise after the exercise finished.
The combined exercises consist of nested maxi and double expressions.
The step vectors are collected in DR.Rules, the student applied all the rules.
When the exercise finished, the success rule fired.
The state was updated. ES.success was set to True, corresponding to exercise state SUCCESS.

In the student model (table \ref{combined.fast.student}) we see all objectives added to the student model.
However EX.double and EX.maxi have also been set.
The fast student rule (section \ref{sec:fststdntrule}) fired after the success rule. It added the prerequisite EX.maxi.
The student now meets the objectives of exercise maxi, so the prerequisites of maxi are added and this added EX.double to the student model.
The rule then fired for the double exercise, but this did not add any new information and a fixpoint is reached.

The effect of this is that exercises double and maxi are no longer recommended.
If the exercises are still presented somewhere in a menu, the student can take one if he wants.



\section{Scenario: A misconception}

\label{sec:misconcept}
\begin{table}[H]
\begin{tabular}{| l | l | l | l |}
\hline
Component & indicator & indicator & value\\
\hline

LT.06.double & ES.objective &EX.double & True\\
 & & PRE.plus &Comprehension\\
 & ES.successCriteria &RL.expr.double & 1\\
 & & RL.expr.plus &1\\
& DR.Rules & RL.expr.double & 1 \\
& DR.Buggy & BUG.buggy.plus & 1\\
& ES.failureCriteria & BUG.buggy.plus  & 1\\
& ES.finished & True & \\
& DR.created & True & \\
& ES.error & True& \\
PT.01.fractions & ES.prerequisite & BUG.buggy.plus & 1\\
 & ES.objective & FIX.buggy.plus & 1 \\
PT.02.fractions & ES.prerequisite & BUG.buggy.plus & 1\\
 & ES.objective & FIX.buggy.plus & 1 \\

\hline
\end{tabular}
\caption{Indicators for exercise after a misconception}
\label{double.buggy.ex}
\end{table}

\begin{table}[H]
\begin{tabular}{| l | l | l | l}
\hline
Component & indicator & value\\
\hline
ES.student & BUG.buggy.plus & 1\\
\hline
\end{tabular}
\caption{Updated student model after exercise combined}
\label{double.buggy.student}
\end{table}


Certain errors are made very often, and this is an indication that the student has a misconception.
An example in our curriculum is working with fractions.
An error that is sometimes made is that the numerators and denominators are added, e.g. ${3\over 4} + {4 \over 5}  = {7 \over 9}$.
This neutralises our doubling operation, e.g.  ${3\over 4} + {3 \over 4}  = {6 \over 8} = {3 \over 4}$.
A domain reasoner can recognise such a mistake with a buggy rewrite rule.

In table \ref{double.buggy.ex} we see the result.
For buggy rules a step vector wil not put the rule execution in DR.Rules but in a vector DR.Buggy.
The vector will also contain a finished indicator, as the exercise stops.
This vector normally should be empty (bottom). 
The buggy inference rule now fires, and adds the DR.Buggy content to the student model (table \ref{double.buggy.student}).

We have anticipated this situation and included some part task exercises to work with fractions.
These exercises have the buggy rule BUG.buggy.plus in their prerequisites.
The recommendation query will now return the part task practice.
We use the same rules for part-task practices and learning tasks. 










\section{Scenario: Variations}
\label{sec:variat}

The \gls{tensteps} recommends variations in exercises.
This prevents a student from getting a one sided view or just learn a trick.
An airline pilot must make landings in head wind and cross wind conditions.
An algebra student with linear equations must make exercises in which alternatively $x$ or $y$ is eliminated.

This places a bit of burden on the exercise designer as we must now keep track of variations.
Let us assume we have maxi exercises with either if-syntax or guards as  a variation.

\begin{table}[H]
\begin{tabular}{| l | l | l | l |}
\hline
Component & indicator & indicator & value\\
\hline
LT.11.maxi.if & ES.objective &EX.maxi1 & True\\
 & & PRE.gtr &Comprehension\\
 & & PRE.if &Comprehension\\
 & ES.prerequisite & EX.double & True\\
 & & & \\
LT.21.maxi.guard & ES.objective &EX.maxi2 & True\\
 & & PRE.gtr &Comprehension\\
 & & PRE.guard &Comprehension\\
 & ES.prerequisite & EX.double & True\\
\hline
LT.33.combined & ES.prerequisite  &EX.maxi1 & True\\
 & ES.prerequisite &EX.maxi2 & True\\

\hline
\end{tabular}
\caption{Exercise maxi with variations}
\label{tab:variations}
\end{table}

Table \ref{tab:variations} shows the prerequisite and objective structure of the maxi variants.
When the student has finished double, both variants are recommended.
When the student completes LT.11, all other instances of the variant (LT.12, LT.13..) 
are no longer recommended, and the student is guided to LT.21, the other variant.

The next exercise (LT.33) specifies both variants as a prerequisite.




\section{Programming tutors}
\label{progtut}

In this chapter we presented scenarios with class 2 exercises.
We have not worked out scenario with a class 3 exercises.
In this section we describe how domain reasoners for class 3 exercises might interact with the model.

Programming tutors such as \gls{askelle} can conduct exercises of class 3.
They do not start with a predefined strategy, but with model solutions for which a strategy is derived.
The set of solutions is not exhaustive. 
The student may arrive at a completely different solution that is nevertheless valid.

A domain reasoner such as  \gls{askelle} can provide procedural support.
If needed the tutor can give a hint or a demonstration.
Ultimately we want the student to execute the exercise without this procedural support.
Using the techniques show before we can guide a student to exercises for which hints are turned off.

A problem is how to update the student model when we cannot count on inferences from  rule executions, as the tutor may not have recognised these in some cases.
In class 2 exercises we often require a student to follow a certain solution strategy (or \gls{sap}).
In that case a constraint based update scheme can be used.
A domain reasoner parses the solution, and provides a step vector with information on solution conditions.
This may include information on a solution correctness, syntax, prelude and language concepts used.
For example, the solutions for the mylength exercise in section \ref{excs:mylength}  listing  \ref{sol2} might provide information on domain concepts.

\begin{table}[H]
\begin{tabular}{| l | l | l |}
\hline
Domain concept & solution nr\\
\hline
prelude: length &  1\\
\hline
syntax: if &  2\\
prelude: null &  2\\
\hline
syntax: pattern matching on lists &  3\\
syntax: where &  3\\
concept: strict evaluation & 3\\
concept: recursion &  3\\
\hline
concept: higher order function  &  4\\
prelude: foldl &  4\\
\hline
\end{tabular}
\caption{Domain concepts in different solutions for listing  \ref{sol2}}
\label{table.cls3}
\end{table}

Table \ref{table.cls3} shows which concepts from functional programming, syntax and prelude were used in different
solutions for the same problem.
These concepts can be put in step vectors and used to update the student model, when the exercise is completed successfully.

The updated student model can be used in a task class to steer the student towards solutions.
In a side bar we can show a list of domain concepts that we want to be  applied.
During solving the exercises the concepts used are removed, and the student must clean the bar. 
If another solution variant is available for an exercise we may refuse a solution that only has familiar concepts.
Some creativity from the instructional designer helps to integrate the student model in user experience design.



\section{Threats to validity}

We have some threats to external validity.


\subsection{External validity}

\subsubsection{Applicability to other domains}
Our approach looks very general and applicable to any domain.
We have used examples from the domain of functional programming.
We have built another domain reasoner for exercises with linear equations.
This is only a limited set of examples.

\subsubsection{Step vector calculation}
Our examples allowed for an easy translation of rules to step vectors.
If the rules correspond to a SAP this is easy.
This may not always be so easy, for example,  sometimes rules are generated automatically.
We do not have a generic procedure for step vector calculation.
This may complicate the design or create extra work for the courseware author.

\subsubsection{Retrofit}
To retrofit the student model on an existing domain reasoner might prove difficult if the educational context is missing.
Exercises must have a well defined \gls{sap}, prerequisites, objectives and level of difficulty.
An extra iteration may be needed and redesign of exercises and solutions.
In the mylength example in Section \ref{excs:mylength} we showed multiple solutions.
Depending on exercise goals one or another solution might be preferable.








