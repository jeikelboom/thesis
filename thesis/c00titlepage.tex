\begin{titlepage}

\newcommand{\HRule}{\rule{\linewidth}{0.5mm}} % Defines a new command for the horizontal lines, change thickness here

\begin{center} % Center everything on the page
 
%----------------------------------------------------------------------------------------
%	HEADING SECTIONS
%----------------------------------------------------------------------------------------

\textsc{\LARGE Open universiteit}\\[1.5cm] % Name of your university/college
\textsc{\Large Management, Science \& Technology}\\[0.5cm] % Major heading such as course name
\textsc{\large Software engineering}\\[1cm] % Minor heading such as course title

%----------------------------------------------------------------------------------------
%	TITLE SECTION
%----------------------------------------------------------------------------------------

\HRule \\[0.4cm]
{ \huge \bfseries  Towards lightweight student modelling \\[0.4cm] for  Functional Programming Tutors}\\[0.4cm] % Title of your document
{ \large  \bfseries Master thesis}\\[0.1cm] % Title of your document
{ \large  \bfseries To be defended  7-nov-2017}\\[0.4cm] % Title of your document
\HRule \\[2.5cm]
 
%----------------------------------------------------------------------------------------
%	AUTHOR SECTION
%----------------------------------------------------------------------------------------

\begin{minipage}{0.49\textwidth}
\begin{flushleft} \large
\emph{Student:}\\
Johan Eikelboom\\ % Your name
835817329\\[1cm]
\emph{Course:}\\
IM9906\\
Master Thesis Software Engineering\\[0.5cm]


\end{flushleft}
\end{minipage}
\begin{minipage}{0.45\textwidth}
\begin{flushright} \large
\emph{Chairman:} \\
prof. dr. Johan Jeuring \\% Supervisor's Name
\emph{Supervisor:} \\
dr. Bastiaan Heeren\\ % Supervisor's Name
\emph{2nd Supervisor:}\\
dr. Lloyd Rutledge
\end{flushright}
\end{minipage}\\[0.5cm]

% If you don't want a supervisor, uncomment the two lines below and remove the section above
%\Large \emph{Author:}\\
%John \textsc{Smith}\\[3cm] % Your name

%----------------------------------------------------------------------------------------
%	DATE SECTION
%----------------------------------------------------------------------------------------

\includegraphics[scale=0.8]{OU_text_logo.pdf}
\end{center}

%----------------------------------------------------------------------------------------
%	LOGO SECTION
%----------------------------------------------------------------------------------------

%\includegraphics{Logo}\\[1cm] % Include a department/university logo - this will require the graphicx package
 
%----------------------------------------------------------------------------------------

%%\vfill % Fill the rest of the page with whitespace

\end{titlepage}
\begin{abstract}
A student model keeps information on student cognitive states, exercise history and personal preferences in Intelligent Tutoring Systems (ITS).
An ITS must select exercises that match student skills, yet are challenging enough to keep attention.
When a student gets stuck or makes mistakes, the ITS must decide on an intervention.
For these tasks the student model must provide inferencing and query capabilities.
The  student model must be updated after each step in an exercise.

We have developed a lightweight approach to student modelling.
This approach contrasts with approaches that use advanced algorithms with high complexity, yet it can make the required inferences with a limited set of rules.
We describe how such a model can be used with domain reasoners from the IDEAS project.
We develop rules for the ``Ten Steps to complex learning'' instructional design methodology.
We illustrate how these rules work in a functional programming curriculum.

\end{abstract}









